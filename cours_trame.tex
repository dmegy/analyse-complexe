\documentclass[11pt,a4paper]{book}

\usepackage[margin=2cm]{geometry}% gestion des marges etc
\usepackage[utf8]{inputenc} % caractères utf8 dans le fichier source
\usepackage[T1]{fontenc} % encodage en sortie
\usepackage[francais]{babel} % paramètres de langue : guillemets etc
\usepackage{amssymb,mathtools,amsthm}
\usepackage{stmaryrd,mathrsfs} % polices et symboles supplémentaires
\usepackage{mdframed,fancybox,graphicx}
\usepackage[dvipsnames]{xcolor}
\definecolor{preuve}{rgb}{0,0.2,0.5}

\usepackage{xypic,multicol,comment,variations,enumerate,enumitem,datetime,tasks}

\usepackage{hyperref}
\hypersetup{
    colorlinks=true,       % false: boxed links; true: colored links
    linkcolor=[rgb]{0.7,0.2,0.2},          % color of internal links
    citecolor=[rgb]{0.7,0.2,0.2},        % color of links to bibliography
    filecolor=[rgb]{0.7,0.2,0.2},      % color of file links
    urlcolor=[rgb]{0.7,0.2,0.2}           % color of external links
}

\usepackage{pgf,pgfmath,tikz}
\usetikzlibrary{arrows}
\usetikzlibrary[patterns]
\tikzset{every picture/.style={execute at begin picture={
   \shorthandoff{:;!?};}
}}




% - - - - - - -
% Spécifique à ce document :
%\usepackage{palatino, euler} % police : Palatino, et Euler pour les maths
\usepackage{fourier} % police de caractères : Adobe Utopia + Fourier math
\everymath{\displaystyle} % plus lisible mais casse l'homogénéité de la mise en page, tant pis la lisiblité passe en premier



\newcommand{\N}{\mathbb{N}}
\newcommand{\Z}{\mathbb{Z}}
\newcommand{\D}{\mathbb{D}}
\newcommand{\Q}{\mathbb{Q}}
\newcommand{\R}{\mathbb{R}}
\newcommand{\C}{\mathbb{C}}
\renewcommand{\H}{\mathbb{H}}
\newcommand{\K}{\mathbb{K}}
\renewcommand{\P}{\mathbb{P}}
\renewcommand{\S}{\mathbb{S}}
\newcommand{\B}{\mathbb{B}}
\newcommand{\U}{\mathbb{U}}


\DeclareMathOperator{\argsh}{argsh}
\DeclareMathOperator{\argch}{argch}

\DeclareMathOperator{\pgcd}{pgcd}
\DeclareMathOperator{\ppcm}{ppcm}
\DeclareMathOperator{\Id}{Id}
\DeclareMathOperator{\Bij}{Bij}
\DeclareMathOperator{\Fix}{Fix}
\DeclareMathOperator{\dist}{dist}
\DeclareMathOperator{\Card}{Card} % cardinal
\renewcommand{\Im}{\operatorname{Im}}
\renewcommand{\Re}{\operatorname{Ré}}
\newcommand{\z}{\overline{z}}
\DeclareMathOperator{\PR}{\text{Ré}}
\DeclareMathOperator{\PI}{\text{Im}}
\DeclareMathOperator{\Log}{\text{Log}}
\renewcommand{\mid}{\;\ifnum\currentgrouptype=16 \middle\fi|\;}
\newcommand\eqdef{\mathrel{\overset{\makebox[0pt]{\mbox{\normalfont\tiny\sffamily déf}}}{=}}}
% égal par définition

\newcommand{\ensemble}[2]{\left \{ #1  
    \ifx&#2&%
       %
    \else%
       \, \middle | \, #2%
    \fi%
\right \}}

\newcommand{\modulo}[1]{\:\left(\operatorname{mod}#1\right)}

% Environnements : 

\theoremstyle{definition}
\newtheorem{theoreme}{Th\'eor\`eme}[section]
\newtheorem{proposition}[theoreme]{Proposition}
\newtheorem{corollaire}[theoreme]{Corollaire}
\newtheorem{lemme}[theoreme]{Lemme}
\renewenvironment{proof}{\color{preuve}\emph{Démonstration.~}}{\qed}
\newenvironment{red}{\begin{quote}\color{preuve}\emph{Exemple de rédaction:}\\}{\end{quote}}

\newtheorem{propdef}[theoreme]{Proposition et Définition}
\newtheorem{axiomedef}[theoreme]{Axiome et Définition}
\newtheorem{definition}[theoreme]{D\'efinition}
\newtheorem{vocabulaire}[theoreme]{Vocabulaire}
\newtheorem{exercice}[theoreme]{Exercice}

\newtheorem{exemple}[theoreme]{Exemple}
\newtheorem{exemples}[theoreme]{Exemples}
\newtheorem{attention}[theoreme]{Mise en garde}
\newtheorem{application}[theoreme]{Application}


\newtheorem{ex}{Exercice}

\theoremstyle{plain}
\newtheorem{remarque}[theoreme]{Remarque}
\newtheorem{methode}[theoreme]{Méthode}




%%%%%%%%%%
% MACROS POUR REMPLACER ANSWERS

%\newenvironment{exo}{\begin{ex}}{\end{ex}}
%\excludecomment{hint} % on n'affiche pas les hints
%\excludecomment{sol} % on n'affiche pas les solutions




% - - - - - - - - - - - - - -
% PARAMETRAGE DU PACKAGE ANSWERS 
% POUR LES INDICATIONS ET CORRECTIONS
% - - - - - - - - - - - - - - 

\usepackage{answers}
\Newassociation{sol}{Soln}{solutions}
% ira dans le fichier d'identifiant 'solutions'
% et écrira les solutions dans un environnement 'Soln'
\newenvironment{exo}{\begin{ex}\label{enonce.\theex} }{\end{ex} }

\renewenvironment{Soln}[1]{\noindent{\bf Correction de l'exercice \ref{enonce.#1}.} }
% - - - - - - - - - - - - - - 
% FIN  PARAMETRAGE ANSWERS
% - - - - - - - - - - - - - - 



\title{Analyse complexe 2023-2024\\ trame de cours}
\author{Damien Mégy}

\begin{document}
\maketitle
\tableofcontents

\chapter{Holomorphie (VIDE)}

\chapter{Fonctions analytiques}

\begin{definition}
Soit $U\subseteq \C$ un ouvert et $f \in \C^U$.
On dit que $f$ est analytique sur $U$ si elle est développable en série entière au voisinage de tout point de $U$.
\end{definition}

Proposition : une série entière de rayon $R$ est analytique sur $\D(0,R)$. Preuve : DSE en $z_0\neq 0$ avec le binôme de Newton.

Définition : un zéro d'une fonction $f$ est un élément $\alpha$ tq $f(\alpha)=0$.


Une série entière identiquement nulle sur un voisinage de l'origine est identiquement nulle sur son disque ouvert de convergence.
Ceci est une conséquence du fait que les coefficients $a_n$ se récupèrent avec les dérivées successives en l'origine. 

Le théorème des zéros isolés est une version un peu plus puissante de ce résultat.

\begin{theoreme}[Zéros isolés pour les séries entières]
Soit $\sum a_nz^n$ une série entière de rayon strictement positif et $f$ sa somme.
S'il existe une suite $(z_n)_{n\in \N} \in (\C^*)^\N$ convergeant vers zéro et telle que $\forall n\in\N, f(z_n)=0$, alors tous les $a_n$ sont nuls.
\end{theoreme}



Définitions : point accumulation d'une partie, point isolé. Partie discrète.

Reformulation du théorème des zéros isolés : si $0$ est un zéro non isolé de $f$, alors $f$ est identiquement nulle.

\begin{theoreme}
Soit $U\subseteq \C$ un ouvert connexe,  $f:U\to\C$ analytique et $Z:=f^{-1}(\{0\})$ l'ensemble des zéros de $f$.
Si $Z$ admet un point d'accumulation dans $U$, alors $f$ est constante.
\end{theoreme}

Preuve 1 : on utilise l'ensemble $A$ des points au voisinage desquels $f$ est identiquement nulle.

Preuve 2 : on utilise l'ensemble $B$ des points où toutes les dérivées de $f$ s'annulent.

\chapter{Intégration curviligne}

\section{Chemins}

Définitions : chemins, lacets, chemin opposé, concaténation de chemins. Chemins $\mathcal C^1$, $\mathcal C^1$ par morceaux.

\section{Formes différentielles}

Formes différentielles réelles et complexes sur un ouvert $U$ de $\C$.

Cas particulier : formes holomorphes.

Opérateur $d : \mathcal C^{\infty}(U,\C) \to \Omega^1(U,\C)$. Image et noyau de cet opérateur. Formes exactes.

\section{Intégrales curvilignes}

Définition pour une forme différentielle.

Invariance par reparamétrage, classe d'équivalence, arcs orientés.

Majoration du module par la norme infinie et la longueur d'arc.

\section{Intégration sur le bord d'un compact régulier}

Rappel : bord d'une partie.

\begin{definition}
Compact à bord $\mathcal C^1$ par morceaux.
\end{definition}

Orientation canonique du bord bien définie.

Exemples à faire soi-même : nombre fini de trous. L'angle à un point anguleux ne peut être nul.  

Remarque : il n'y a qu'un nombre fini de points anguleux sur le bord. Le bord est paramétrable par un nombre fini d'arcs $\mathcal C^1$.

\begin{definition}
Intégration d'une $1$-forme différentielle de classe $\mathcal C^1$ sur le bord d'un compact à bord $C^1$ par morceaux.
\end{definition}

\section{Primitives}

Attention aux différentes significations du mot \og primitive\fg. 

\begin{definition}[Primitive (holomorphe, d'une fonction)]
On dit que $f : U\subset \C\to \C$ admet une primitive (holomorphe,) ou simplement une primitive, s'il existe $F \in \mathcal O(U)$ telle que $F'=f$.
\end{definition}


Attention, certaines fonctions holomorphes n'ont pas de primitive, contrairement à ce qui se passe en analyse réelle où toute fonction continue admet une primitive. 
Par exemple la fonction $f:\C^*\to \C, z\mapsto 1/z$ n'a pas de primitive sur $\C^*$. 

Le fait d'avoir une primitive dépend de façon cruciale de l'ouvert de définition. Par exemple la fonction $g : \C\setminus \R_- \to \C, z\mapsto 1/z$ possède une primitive, à savoir la détermination principale du log, notée $\operatorname{Log}$ dans ce cours (ou $\operatorname{Log}_0$, aussi).


\begin{proposition}
Si $f$ admet une primitive holomorphe, l'intégrale curviligne $\int_\gamma f(z)dz$ ne dépend que des extrémités du chemin.
\end{proposition}

Dans ce qui suit, le mot \og primitive\fg{} a encore un nouveau sens, il s'applique cette fois à des formes différentielles et plus à des fonctions.

Vocabulaire : si $\omega$ est une $1$-forme exacte,  c'est-à-dire qu'il existe $f$ telle que $\omega=df$, alors on dit aussi qu'elle admet une primitive. (La primitive est la fonction $f$.)

Attention dans ce nouveau contexte, la primitive de la forme différentielle n'est pas forcément  une fonction holomorphe, c'est juste une fonction $C^1$.

Ce sens généralise le sens précédent comme suit: la fonction $f$ possède une primitive holomorphe si la forme différentielle $f(z)dz$ possède une primitive au sens des formes différentielles. (Exercice : il s'agit de montrer que dans ce cas la primitive est automatiquement holomorphe.)

\begin{proposition}
Soit $\omega$ une $1$-forme différentielle.
Si $\omega$ admet une primitive, l'intégrale curviligne  $\int_\gamma \omega$ ne dépend que des extrémités du chemin $\gamma$.
\end{proposition}


\subsection{Convergence}

Laissé en exercice : 

intégration sur un chemin fixé d'une suite de fonctions qui CVU (ou CVUTC).

Cas plus général : que faire si on n'a pas CVUTC ? dans certaines conditions, écrire explicitement le paramétrage et convergence dominée.

intégration d'une forme donnée sur une suite de chemins qui converge en topologie $C^1$ vers un chemin $\mathcal C^1$ par morceaux.




\chapter{Théorème et formule(s) de Cauchy}

\section{Théorème intégral de Cauchy}

\begin{theoreme}
Soit $U\subset \C$ un ouvert, $f\in \mathcal O(U)$, et $K\subseteq U$ un compact à bords $\mathcal C^1$ par morceaux, dont le bord est muni de son orientation canonique.
Alors, $\int_{\partial K} f(z)dz = 0$.
\end{theoreme}

\begin{attention}
\begin{enumerate}
\item Le bord n'est pas forcément connexe. Par exemple, le bord d'une couronne fermée est l'union de deux cercles.
\item Attention à l'orientation du bord, en particulier lorsqu'il n'est pas connexe comme dans l'exemple précédent.
\end{enumerate}
\end{attention}

Historique : Cauchy, Riemann, Goursat.

Plan de la preuve : 
\begin{enumerate}
\item Lemme de Goursat.
\item Approximation polygonale du compact. Ceci n'a pas été démontré en détail et ne fera pas l'objet de questions de cours.
\item Fin de la preuve : approximation de l'intégrale curviligne par les intégrales sur les triangulations de plus en plus fines.  Ceci n'a pas été démontré en détail et ne fera pas l'objet de questions de cours.
\end{enumerate}

%Évocation des autres versions trouvable dans les livres, avertissements.





\section{Formule de Cauchy}

Formule de Cauchy
% en évidant un petit disque dans le compact et en passant à la limite, pas la preuve dégueulasse du Tauvel.

%(Faire la preuve de Damian plus tard en commentaire. Faire également la preuve de la formule avec l'indice plus tard, une fois les déf de base sur l'indices faites, et qq exos faits.)

Cas particulier avec des cercles.

Conséquence : 

\begin{theoreme}
Les fonctions holomorphes sont analytiques.
De plus, si $f\in \mathcal O(U)$ et $z_0 \in U$, le rayon de convergence du DSE de $f$ en $z_0$ a un rayon supérieur ou égal à $\operatorname{dist}(z_0,U^c)$.
\end{theoreme}

Conséquence : 
\begin{enumerate}
\item Les fonctions holomorphes sont $\C^\infty$.
\item Tous les théorèmes vrais pour les fonctions analytiques sont vrais pour les fonctions holomorphes : zéros isolés, prolongement analytique etc.
\end{enumerate}

Dans le cas particulier de cercles centrés, la formule de Cauchy se spécialise en la formule suivante, appelée formule de la moyenne : 
\begin{theoreme}[Formule de la moyenne]
Soit $U\subseteq \C$ un ouvert,  $f \in \mathcal O(U)$ et $z_0\in U$.
Pour tout $r>0$ tel que $\overline{\D(z_0,r} \subseteq U$, on a :
\[ f(z_0) = \frac{1}{2\pi}\int_0^{2\pi} f\left(z_0+re^{it}\right)dt\]
\end{theoreme}
\emph{(Interprétation : la valeur de $f$ en $z_0$ est la moyenne de ses valeurs sur un cercle centré en $z_0$. 
Preuve : c'est juste la formule de Cauchy. 
Erreur de compréhension possible : la formule ne dit PAS que $f(z_0)=\int_{\mathcal C(z_0,r} f(z)dz$.)}


\section{Formules de Cauchy d'ordre supérieur}
 
\begin{theoreme}
Soit $U\subseteq \C$ un ouvert et $f\in \mathcal O(U)$.
Soit $K\subseteq U$ un compact à bords $C^1$ par morceaux et $z\in K^\circ$.
Alors, pour tout $n\geq 0$ :
\[ \frac{f^{(n)}(z)}{n!} = \frac{1}{2i\pi} \int_{\partial K} \frac{f(w)}{(w-z)^{n+1}}dw\]
\end{theoreme}
Remarque : pour $n=0$, c'est la formule de Cauchy classique.

 Cas particulier des cercles centrés : on récupère les formules intégrales pour les coefficients des séries entières.
 
 En prenant les formules de Cauchy et en appliquant l'inégalité triangulaire, on obtient ce qu'on appelle les inégalités de Cauchy.
 Dans le cas particulier de cercles centrés, on obtient les formules suivantes très souvent utiles :
 
\begin{corollaire}[Inégalités de Cauchy pour cercles centrés]
Soit $U\subseteq \C$ un ouvert,  $f\in \mathcal O(U)$ et $\overline{\mathbb D(z,r)}$ un disque fermé inclus dans $U$.
Alors,  pour tout $n\geq 0$ :
\[ \frac{\left|f^{(n)}(z)\right|}{n!} \leq \frac{1}{r^n} \max_{w\in \mathcal C(z,r)} |f(w)|\]
\end{corollaire}
 

On étudie maintenant les propriétés locales des fonctions holomorphes.
On commence par une conséquence directe de la formule de la moyenne : une fonction holomorphe admettant un maximum local (en module) en un point est localement constante au voisinage de ce point.
 
\begin{theoreme}[Principe du maximum]
Soit $U\subseteq \C$ un ouvert et $f \in \mathcal O(U)$.
Si $|f|$ admet un maximum local en un point $z_0$, alors $f$ est constante sur la composante connexe de de $U$ contenant $z_0$.
\end{theoreme}

Variations, voir feuilles de TD:
\begin{enumerate}
\item Sur un compact $K\subset U$, le module atteint son max au bord de $K$.
\item Principe dit \og du minimum\fg.
\end{enumerate}
 
 \begin{theoreme}[de Liouville]
 Soit $f$ une fonction holomorphe sur $\C$.
 Si $f$ est bornée, alors $f$ est constante.
 \end{theoreme}
 
 Avertissement : il est important que la fonction soit holomorphe sur $\C$ tout entier. 
 Si vous voulez appliquer Liouville, vous l'appliquez à une fonction holomorphe sur $\C$ (et bornée), rien d'autre.  
Si vous avez une fonction définie sur un ouvert plus petit que $\C$ et que vous souhaitez montrer qu'elle est constante, alors soit vous montrez préalablement qu'elle peut s'étendre en fonction holomorphe sur $\C$ tout entier (ça n'est pas forcément le cas) et vous appliquez Liouville classique, soit vous faites autrement.



 
\begin{corollaire}[Théorème de d'Alembert-Gauss]
Tout polynôme complexe non constant admet une racine sur $\C$.
\end{corollaire} 

Preuve : supposons par l'absurde qu'un polynôme $P$ non constant ne s'annule pas. Alors la fonction $f=1/P$ est entière et bornée, donc constante, contradiction.
 

% Formule de Pompeiu : NON


\section{Étude locale}

\begin{theoreme}[Inversion locale]
Soit $u\subseteq \C$ un ouvert, $f\in \mathcal O(U)$ et $z_0\in U$ un point tel que $f'(z_0)\neq 0$.
Alors il existe un ouvert $V$ contenant $z_0$ tel que $W:=f(V)$ soit ouvert et que $f|_V$ soit un biholomorphisme de $V$ sur $W$.
\end{theoreme}

%Preuve donnée en cours : la différentielle d'une fonction holomorphe est nulle ou inversible.  Vu les hypothèses du théorème, cette différentielle est injective en $z_0$. On applique alors le théorème d'inversion locale vu en calcul différentiel qui donne les ouverts $V$ et $W$ sur lesquels $f$ réalise un $\mathcal C^1$-difféomorphisme. 
%Enfin, si $v\in V$ et $w:=f(v)$, la différentielle en $w$ de l'application  $\left(f|_W\right)^{-1}$ est l'inverse de $Df_v$.
%Comme une matrice $\C$-linéaire on nulle admet un inverse $\C$-linéaire, on en déduit que $\left(f|_W\right)^{-1}$ est holomorphe sur $W$.

%Remarque : il y a une preuve directe qui n'utilise pas l'énoncé démontré en calcul différentiel, à l'aide de ce qu'on appelle des \emph{séries majorantes}. Elle est plus longue mais a l'avantage d'être autocontenue et de donner plus d'informations sur les ouverts $V$ et $W$.

Multiplicité d'un zéro, 
Point critique, multiplicité d'un point critique

Étude aux points critiques : énoncé, pas démontré.

[Rappel de terminologie : une fonction continue est dite ouverte si l'image de tout ouvert est ouverte.]

\begin{theoreme}
Soit $U$ un ouvert connexe et $f\in \mathcal O(U)$ non constante.
Alors, $f$ est ouverte.
\end{theoreme}

\begin{center}
\hrulefill \emph{Fin du cours du 27 mars} \hrulefill
\end{center}

Preuve de l'étude au point critique : détermination de racines $n$-èmes et de logarithmes complexes, puis preuve.

Preuve de l'application ouverte : compléter ici.



\begin{application}
Tous les exos du type \og une fonction holomorphe sur un connexe à valeurs dans $\R$ est constante\fg{} sont maintenant faisables avec l'application ouverte, sans devoir calculer avec Cauchy-Riemann.
\end{application}

\begin{application}
Deuxième preuve du principe du maximum : si $z_0 \in U$ et que $f$ n'est pas localement constante au voisinage de $z_0$, alors $f(U)$ contient un voisinage de $f(z_0)$ et donc $|f|$ n'atteint pas son maximum en $z_0$.
\end{application}


\begin{theoreme}[Lemme de Schwarz : vu en TD]
\end{theoreme}

Attention à la preuve, il se produit quelque chose d'assez surprenant dans la majoration : en majorant sur un ensemble plus grand,  on arrive à obtenir un majorant plus petit. Ceci est un effet du principe du maximum et n'est absolument pas intuitif.



\begin{theoreme}[Biholomorphismes du disque : compléter la preuve]
\end{theoreme}


%
%\section{Non traité  en CM : intégrales dépendant holomorphiquement d'un paramètre}
%
%Intégrales dépendant holomorphiquement d'un paramètre.
%
%TD : fonction Gamma d'Euler
%
%
%\section{Non traité en CM : primitives, théorème de Morera}
%
%\begin{theoreme}[Morera]
%Soit $\Omega$ un ouvert et $f \in C^0(\Omega)$.
%Si $f(z)dz$ a une intégrale nulle sur tout bord de triangle de $\Omega$, alors $f$ est holomorphe.
%\end{theoreme}
%
%Preuve : on prend un point, un disque autour et on construit une primitive locale holomorphe sur ce disque, qui est étoilé en $z_0$.
%
%Attention ne pas dériver en $z_0$ mais en $z$ dans le disque.
%
%Remarque  : version un peu plus générale : si f est continue et d'intégrale curviligne nulle sur tout triangle sur un ouvert étoilé, elle a une primitive sur cet ouvert étoilé et doc est holomorphe sur cet ouverte étoilé.
%
%A rapprocher des énoncés : si sur un ouvert connexe l'intégrale ets nulle sur tout lacet, on a une primitive.
%

%\section{Topologie des espaces de fonctions holomorphes}

%Non, sauter entièrement, tant pis.

% mettre l'exo sur une limite uniforme d'holomorphes





\chapter{Séries de Laurent, singularités isolées, Résidus}

\section{Séries de Laurent}

Séries de Laurent, exemples.
Partie régulière, partie principale/polaire. 


Couronnes, notations, couronne ouverte de convergence d'une série de Laurent, convergence normale sur sous-couronne fermée de la couronne ouverte de convergence.

Dérivée d'une série de Laurent. 

\begin{theoreme}[Développement en série de Laurent des fonctions holomorphes sur une couronne]
Une fonction holomorphe sur une couronne est développable en série de Laurent sur cette couronne et $a_n = \cdots$.
\end{theoreme}


Exercice : Développer $\frac{1}{z-1}$ en série de Laurent sur la couronne $C(0,2,3)$. (Revoir le TD.)


\section{Singularités isolées}


Définition : singularité isolée.

Exemples et contre-exemples. L'origine n'est pas une singularité de $\Log_0$, ni de $\sin(1/z)^{-1}$.

\subsection{Singularités effaçables}

\begin{definition}
Une singularités isolée $z_0$ d'une fonction holomorphe $f$ est dite \emph{effaçable} si $f$ se prolonge holomorphiquement en $z_0$.
\end{definition}

Rq : on lit aussi \og éliminable\fg{} au lieu d'effaçable dans certains ouvrages. Dans ce cours, on adopte néanmoins la terminologie \og singularité effaçable\fg, qui est assez standard.

\begin{proposition}
Soit $z_0$ une singularités isolée d'une fonction holomorphe $f$.
Soit $\sum_{n\in \Z} a_n(z-z_0)^n$ le développement en série de Laurent de $f$ sur un disque épointé $\D(z_0,\epsilon)^*$.
Alors, $z_0$ est une singularité effaçable ssi $\forall n<0, a_n=0$.
\end{proposition}

Idée de preuve : pour le sens difficile, si $f$ se prolonge en une fonction holomorphe $\tilde f$, alors on développe $\tilde f$ en série entière, et on invoque l'unicité des coefficients d'un développement en série de Laurent sur le disque épointé.

\begin{theoreme}[Théorème d'extension de Riemann]
Soit $z_0$ une singularité isolée d'une fonction holomorphe $f$.
Si $|f|$ est bornée au voisinage de $z_0$, alors la singularité est effaçable, autrement dit $f$ se prolonge holomorphiquement en $z_0$ !
\end{theoreme}

\begin{remarque}
Encore un résultat spectaculaire (évoqué lors du tout premier cours), à l'opposé de ce qu'il se passe  en analyse réelle classique.
Une fonction de $\mathcal C^\infty(\R^*,\R)$ (et même analytique réelle sur $\R^*$), bornée au voisinage de l'origine, n'a en effet aucune raison de se prolonger ne serait-ce que continûment en l'origine. 
Dans le cadre holomorphe, être localement bornée autour d'une singularité isolée garantit un prolongement holomorphe (en particulier $\mathcal C^\infty$ mais bien plus) !
\end{remarque}


\begin{center}
\hrulefill \emph{Fin du cours du 10 avril} \hrulefill
\end{center}

\subsection{Cas non effaçable : pôles et singularités essentielles}

\begin{definition}
Soit $z_0$ une singularité isolée non effaçable d'une fonction holomorphe $f$.
Soit $\sum_{n\in \Z}a_n(z-z_0)^n$ le développement en série de Laurent de $f$ sur un disque épointé centré sur $z_0$.
On dit que $z_0$ est:
\begin{itemize}
\item un \emph{pôle} si $\{n<0\:|\: a_n\neq 0\}$ est fini. (Nombre fini de termes dans la partie principale.) Dans ce cas, $m:=\min\{n\in \Z\:|\: a_n\neq 0\}$ est l'ordre du pôle.
\item une singularité essentielle sinon. (Infinité de termes dans la partie principale.)
\end{itemize}
\end{definition}

\begin{exemples}
L'origine $0$ est un pôle de $1/z^3$, d'ordre $3$. 
L'origine $0$ est une singularité essentielle de $z\mapsto \exp(1/z)=\sum_{p\geq 0}\frac{(1/z)^p}{p!} = \sum_{n\leq 0} \frac{z^n}{(-n)!}$.
\end{exemples}

\begin{proposition}[Comportement au voisinage d'un pôle]
\end{proposition}

\begin{theoreme}[Casoratti-Weiestrass]
Soit $z_0 \in U$ une singularité essentielle d'une fonction holomorphe $f \in \mathcal O(U\setminus \{z_0\})$.
Alors, pour tout voisinage $W$ de $z_0$ contenu dans $U$, l'image par $f$ de $W\setminus\{z_0\}$ est dense dans $\C$.
\end{theoreme}

Moralité : le comportement au voisinage d'une singularité essentielle est extrêmement pathologique, bien pire que simplement tendre vers $+\infty$ en module. Un bon exemple est $\exp{1/z}$ qui fait des choses fondamentalement différentes suivant comment on s'approche de l'origine.

\section{Fonctions méromorphes}

\begin{proposition}
Soit $U$ un ouvert,  $Z\subset U$ un fermé constitué de points isolés et $f$ fonction holomorphe sur $U\setminus Z$.
Alors il y a équivalence entre :
\begin{itemize}
\item Pour tout $z_0\in Z$, $z_0$ n'est pas une singularité essentielle de $f$.
\item Au voisinage de tout $z_0\in Z$, $f$ s'écrit comme le quotient $g/h$ de deux fonctions holomorphes (définies sur un voisinage de $z_0$).
\end{itemize}
\end{proposition}

\begin{definition}
Soit $U$ un ouvert.
Une fonction méromorphe sur $U$ est la donnée d'un fermé $Z\subset U$ constitué de points isolés, et d'une fonction holomorphe sur $U\setminus Z$ vérifiant, pour tout $z_0\in Z$, les conditions équivalentes de la proposition précédente.
\end{definition}

Rq : la seconde définition se réinterprète en disant que $f$ s'étend en une application  de $U$ vers la droite projective complexe.


Attention, une \og fonction méromorphe sur $U$\fg{} n'est donc PAS une fonction de $U$ dans $\C$. Par exemple, on dit que $z\mapsto \frac{1}{z}$ est méromorphe sur $\C$ tout entier.
Ca ne veut pas dire que c'est une fonction définie sur $\C$, ça veut juste dire ce qu'il y a dans la définition.

%Calculs avec les ordres des zéros et pôles.


\section{Théorème des résidus}

\begin{definition}
Résidus.
\end{definition}

Remarque : le résidu est défini même pour une singularité essentielle.

Techniques de calcul de calcul de résidus.

Attention : si vous travaillez sur d'autres ouvrages, vous croiserez \emph{beaucoup} d'énoncés tous un peu différents sur les résidus. Formules homotopiques, homologiques, avec l'intervention d'indices, des conditions de simple connexité  etc... Nous n'avons pas vu ces notions. 

La forme donnée ici est la plus basique et efficace, elle ne nécessite pas de connaissances en topologie algébrique et s'applique dans un très grand nombre de cas en pratique. 
(Son désavantage est bien sûr qu'elle n'aborde pas les thématiques topologiques évoquées plus haut, parce qu'elle les évite. C'est dommage, mais cette année nous n'aurons pas le temps de traiter ces thématiques.)

\begin{theoreme}[des résidus]
Soit $U\subset \C*$ un ouvert, $Z\subset U$ une partie discrète, $K\subset U$ un compact à bords $\mathcal C^1$ par morceaux avec $Z\cap \partial K = \emptyset$, et $f$ une fonction holomorphe sur $U\setminus Z$.
Alors:
\[ \int_{\partial K} f(z)dz = 2i\pi \sum_{\alpha \in K\cap Z} \operatorname{Rés}_\alpha(f).\]
\end{theoreme}

Techniques de calculs de résidus, en particulier pour pôles simples et doubles : pour un pôle d'ordre $m$, calcul du résidu soit en faisant un DL, soit avec la formule de Taylor pour les coefficients (dérivées successives).


Applications : calculs d'intégrales réelles. Cas traités en CM : $\int_\R \frac{dx}{1+x^2}$ et $\int_\R \frac{dx}{1+x^4}$.

Feuille de TD8 en ligne, avec quelques exemples corrigés.

\begin{center}
\hrulefill \emph{Fin du cours du 17 avril} \hrulefill
\end{center}

\end{document}


Progression Damian : 
Goursat : holomorphe implique intégrale nulle sur tout triangle.
Existence de primitives pour les fonctions holomorphes sur un ouvert étoilé.
Théorème de Cauchy sur un ouvert étoilé.
Formule de Cauchy sur un disque, en coupant en deux.
Formules de Cauchy d'ordre un, puis $n$ (admis)
Morera
Analycité des fonctions holomorphes, et résultat sur le rayon de convergence
Variations sur Cauchy : formule de la moyenne, inégalités de Cauchy, Liouville, amélioration polynomiale, théorème fondamental de l'algèbre. (avec... preuve analytique!), théorème de l'image ouverte, principe du maximum

à pousser dans une autre chapitre : convergence de suites de fonctions holomorphes, intégrales à paramètres.

Attention il y a l'extension de Riemann en exo ?? ainsi que le principe de réflexion de Schwarz, et le lemme de Schwarz etc.

Damian a un chapitre 5 Cauchy Généralisé où il met ensuite:
homotopies, simple connexité, , théorème de Cauchy pour un lacet dans un simplement connexe, Cauchy homotopique, conséquence $\C^*$ n'est pas simplement connexe. ENsuite : indice, formules de Cauchy homotopiques, qui commencent vraiment à ressembler à des formules de résidus. ENsuite : homologie : chaînes, cycles etc. Formule de Cauchy homologique. 
Logarithmes, logarithme d'une fonction holomorphe non nulle sur un ouvert simplement connexe..

Tout ça sans encore avoir fait les singularités, les séries de Laurent etc etc





\section{Progression Audin}


\section{Progression Demailly}

(Note historique : Théorème de Cauchy : énoncé par Cauchy en 1825 sans preuve rigoureuse, démontré par Riemann en 1851 dans le cas $\mathcal C^1$, puis démontré par Goursat en ...1900 ! )

Lemme de Goursat
Théorème de Goursat = théorème de Cauchy sur le bord d'un compact régulier
Formule de Cauchy pour le bord dun compact inclus dans U.
Preuve en triangulant.


Plus rapide que la progression de Damian, on a presque la version homologique, ou du moins la partie utile de la version homologique.

Ensuite : il se paye de l luxe de faire la formule de Pompeiu, puis : conséquences : infiniment dérivable au sens complexe (en calculant les dérivées partielles en z et \z.
Morera
Inégalités de Cauchy et fonctions à croissance polynomiale, Liouville, d'Alembert-Gauss
Analycité réelle, et holomorphe implique C-analytique.
Corollaire important sur le rayon de convergence ! (3.3, p.41)
Conséquences : zéros des fonctions holomorphes 
zéros isolés (maison a déj) montré ça), ordre d'un zéro, prolongement analytique 'déjà pour fonctions analytiques)
Prolongement à la frontière !
Inversion locale holomorphe : une preuve qui utilise l'inversion locale C^\infty, et l'autre, la méthode des séries majorantes (ne pas faire)
application ouverte
" inversion globale"  : une fonction holomorphe injective est un biholmorphisme sur son image. Attention à ne pas confondre avec l'inversion globale en calcul diff
Principe du max, lemme de SChwarz, automorphismes du disque.
Intégrales dépendant d'un paramètre, fonction Gamma,
Topologie des espaces de fonctions holomorphes, produits infinis, familles normales, Montel.
Seulement ensuite, points singuliers, fonctions méromorphes, résidus.
ENcore ensuite : topo alg : homotopies, lacets, homologie etc. Indice, Rouché, Runge.
APrès ça part sur les sous-harmoniques


\section{Progression Carlier}
Indice, definition et valeur entière, nulle sur la composante non bornée du complémentaire. Illustration, nombre de tours, non démontré.
Primitives complexes : définition, admettre une primitive localement, globalement.
Prop : existence d'une primitive \iff intégrale nulle sur tout lacet
Prop : U convexe (ou étoilé) : f admet une primitive ssi pour tout triangle inclus dans U, l'intégrale sur le triangle est nulle.
Lemme de Goursait, lemme de Goursat amélioré
Cor : f holomorphe (ou sauf en un pt où elle est continue), alors $f$ admet une primitive locale. SI l'ouvert est convexe (remplacer par étoilé), alors prmitive globale.
Th : holomorphe sur un étoilé => primitive globale
Formule de Cauchy, pour un lacet à l'intérieur d'un disque sur lequel la fonction est holomorphe, attention, et avec l'indice qui sort. Preuve avec le Goursat amélioré qui donne une primitive locale, donc la nullité d'une intégrale, mais il faut comprendre l'indice et la formule reste relativement locale. Note : on pourrait remplacer le disque par un ouvert étoilé. Ca donne Cauchy avec indice, pour $f$ holomorphe sur un ouvert étoilé. Au fond c'est pareil que Damian. Un peu plus général car indice, mais pas d'interprétation prouvée de l'indice.
Note : en fait il a le théorème de Cauchy sur un ouvert étoilé, mais il ne l'énonce même pas ? pas vu.
Forme simplifiée pour un cercle
conséquence : $\C$-analycité.
Attention son coef a_n dépend de $r$, il ne justifie/remarque pas que ça ne dépend pas.
Corollaire : inégalités de Cauchy









