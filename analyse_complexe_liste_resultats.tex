\documentclass[11pt,a4paper]{article}

\usepackage[margin=2cm]{geometry}% gestion des marges etc
\usepackage[utf8]{inputenc} % caractères utf8 dans le fichier source
\usepackage[T1]{fontenc} % encodage en sortie
\usepackage[francais]{babel} % paramètres de langue : guillemets etc
\usepackage{amssymb,mathtools,amsthm}
\usepackage{stmaryrd,mathrsfs} % polices et symboles supplémentaires
\usepackage{mdframed,fancybox,graphicx}
\usepackage[dvipsnames]{xcolor}
\definecolor{preuve}{rgb}{0,0.2,0.5}

\usepackage{xypic,multicol,comment,variations,enumerate,enumitem,datetime,tasks}

\usepackage{hyperref}
\hypersetup{
    colorlinks=true,       % false: boxed links; true: colored links
    linkcolor=[rgb]{0.7,0.2,0.2},          % color of internal links
    citecolor=[rgb]{0.7,0.2,0.2},        % color of links to bibliography
    filecolor=[rgb]{0.7,0.2,0.2},      % color of file links
    urlcolor=[rgb]{0.7,0.2,0.2}           % color of external links
}

\usepackage{pgf,pgfmath,tikz}
\usetikzlibrary{arrows}
\usetikzlibrary[patterns]
\tikzset{every picture/.style={execute at begin picture={
   \shorthandoff{:;!?};}
}}




% - - - - - - -
% Spécifique à ce document :
%\usepackage{palatino, euler} % police : Palatino, et Euler pour les maths
\usepackage{fourier} % police de caractères : Adobe Utopia + Fourier math
\everymath{\displaystyle} % plus lisible mais casse l'homogénéité de la mise en page, tant pis la lisiblité passe en premier



\newcommand{\N}{\mathbb{N}}
\newcommand{\Z}{\mathbb{Z}}
\newcommand{\D}{\mathbb{D}}
\newcommand{\Q}{\mathbb{Q}}
\newcommand{\R}{\mathbb{R}}
\newcommand{\C}{\mathbb{C}}
\renewcommand{\H}{\mathbb{H}}
\newcommand{\K}{\mathbb{K}}
\renewcommand{\P}{\mathbb{P}}
\renewcommand{\S}{\mathbb{S}}
\newcommand{\B}{\mathbb{B}}
\newcommand{\U}{\mathbb{U}}


\DeclareMathOperator{\argsh}{argsh}
\DeclareMathOperator{\argch}{argch}

\DeclareMathOperator{\supp}{supp}
\DeclareMathOperator{\pgcd}{pgcd}
\DeclareMathOperator{\ppcm}{ppcm}
\DeclareMathOperator{\Id}{Id}
\DeclareMathOperator{\Bij}{Bij}
\DeclareMathOperator{\Fix}{Fix}
\DeclareMathOperator{\dist}{dist}
\DeclareMathOperator{\Card}{Card} % cardinal
\renewcommand{\Im}{\operatorname{Im}}
\renewcommand{\Re}{\operatorname{Ré}}
\newcommand{\z}{\overline{z}}
\DeclareMathOperator{\PR}{\text{Ré}}
\DeclareMathOperator{\PI}{\text{Im}}
\DeclareMathOperator{\Log}{\text{Log}}
\renewcommand{\mid}{\;\ifnum\currentgrouptype=16 \middle\fi|\;}
\newcommand\eqdef{\mathrel{\overset{\makebox[0pt]{\mbox{\normalfont\tiny\sffamily déf}}}{=}}}
% égal par définition

\delimitershortfall-1sp
\newcommand\abs[1]{\left|#1\right|}

\newcommand{\ensemble}[2]{\left \{ #1  
    \ifx&#2&%
       %
    \else%
       \, \middle | \, #2%
    \fi%
\right \}}

\newcommand{\modulo}[1]{\:\left(\operatorname{mod}#1\right)}

% Environnements : 

\theoremstyle{definition}
\newtheorem{theoreme}{Th\'eor\`eme}[section]
\newtheorem{proposition}[theoreme]{Proposition}
\newtheorem{prop}[theoreme]{Proposition}
\newtheorem{corollaire}[theoreme]{Corollaire}
\newtheorem{lemme}[theoreme]{Lemme}
\renewenvironment{proof}{\color{preuve}\emph{Démonstration.~}}{\qed}
\newenvironment{red}{\begin{quote}\color{preuve}\emph{Exemple de rédaction:}\\}{\end{quote}}

\newtheorem{propdef}[theoreme]{Proposition et Définition}
\newtheorem{axiomedef}[theoreme]{Axiome et Définition}
\newtheorem{definition}[theoreme]{D\'efinition}
\newtheorem{vocabulaire}[theoreme]{Vocabulaire}
\newtheorem{exercice}[theoreme]{Exercice}

\newtheorem{exemple}[theoreme]{Exemple}
\newtheorem{exemples}[theoreme]{Exemples}
\newtheorem{attention}[theoreme]{Mise en garde}
\newtheorem{application}[theoreme]{Application}


\newtheorem{ex}{Exercice}

\theoremstyle{plain}
\newtheorem{remarque}[theoreme]{Remarque}
\newtheorem{methode}[theoreme]{Méthode}




%%%%%%%%%%
% MACROS POUR REMPLACER ANSWERS

%\newenvironment{exo}{\begin{ex}}{\end{ex}}
%\excludecomment{hint} % on n'affiche pas les hints
%\excludecomment{sol} % on n'affiche pas les solutions




% - - - - - - - - - - - - - -
% PARAMETRAGE DU PACKAGE ANSWERS 
% POUR LES INDICATIONS ET CORRECTIONS
% - - - - - - - - - - - - - - 

\usepackage{answers}
\Newassociation{sol}{Soln}{solutions}
% ira dans le fichier d'identifiant 'solutions'
% et écrira les solutions dans un environnement 'Soln'
\newenvironment{exo}{\begin{ex}\label{enonce.\theex} }{\end{ex} }

\renewenvironment{Soln}[1]{\noindent{\bf Correction de l'exercice \ref{enonce.#1}.} }
% - - - - - - - - - - - - - - 
% FIN  PARAMETRAGE ANSWERS
% - - - - - - - - - - - - - - 



\title{Analyse complexe 2024-2025\\ Aide aux révisions : liste de résultats}
\author{Damien Mégy}

\begin{document}
\maketitle
\tableofcontents

\section{Holomorphie}

\begin{prop}
Toute matrice $M\in M_2(\R)$ est la somme d'une matrice $\C$-linéaire et d'une matrice $\C$-antilinéaire et cette décomposition est unique.
\end{prop}

\begin{prop}
Une matrice de similitude directe est soit nulle soit inversible, auquel cas son inverse est une similitude directe.
\end{prop}

\begin{prop}[Conditions de Cauchy-Riemann]
Une fonction $f:U\to \C$ est $\C$-dérivable en un point $z_0 \in U$ si et seulement si elle est différentiable en $z_0$ et de plus, sa différentielle en $z_0$ est une similitude directe.
\end{prop}

\begin{prop}
Toutes les règles de dérivation pour les opérateurs de Wirtinger.
\end{prop}
\section{Fonctions analytiques}

\subsection{Séries formelles}

\begin{prop}
Les inversibles de $k[[t]]$ sont les éléments $\sum a_kT^k$ avec $a_0\neq 0$.
\end{prop}

\subsection{Séries entières}


\begin{prop}
Soit $\sum a_nz^n$ une série entière.
Son rayon de convergence est strictement positif  ssi il existe $C>0$ tel que $a_n = O\left(C^n\right)$.
\end{prop}

\begin{theoreme}[Hadamard-Cauchy]
Soit $\sum a_nz^n$ une série entière, et soit $\ell = \limsup \sqrt[n]{|a_n|} \in \R_+\cup \{+\infty\}$.
Alors le rayon de convergence de la série est $1/\ell \in \R_+\cup \{+\infty\}$.
\end{theoreme}

\begin{prop}[Convergence des produits de Cauchy]
Soient $\sum a_nz^n$ et $\sum b_nz^n$ deux séries entières de rayons $R>0$ et $R'>0$.
Alors le produit de Cauchy $\sum c_nz^n$ des deux séries a un rayon de convergence $R'\geq \min(R,R')$.
\end{prop}

\begin{prop}
Soit $\sum a_nz^n$ une série entière non nulle de rayon $R>0$ et $f$ sa somme sur le disque $D(0,R)$.
Soit $d$ l'indice du premier coefficient non nul de $\sum a_nz^n$.
Alors, au voisinage de zéro, on a  $f(z)\sim a_dz^d$.
\end{prop}

\begin{prop}
Soit $\sum a_nz^n$ une série entière de rayon $R>0$ et $f$ sa somme sur le disque $D(0,R)$.
Alors $f$ est dérivable au sens complexe sur $D(0,R)$ et sa dérivée complexe est la somme de la série entière $\sum (n+1)a_{n+1}z^n$ (qui est aussi de rayon $R$).
\end{prop}

\begin{theoreme}[Zéros isolés pour les séries entières, à l'origine]
Soit $\sum a_nz^n$ une série entière de rayon strictement positif et $f$ sa somme.
S'il existe une suite $(z_n)_{n\in \N} \in (\C^*)^\N$ convergeant vers zéro et telle que $\forall n\in\N, f(z_n)=0$, alors tous les $a_n$ sont nuls.
\end{theoreme}

\begin{theoreme}[Principe du maximum pour une série entière, en l'origine]
Soit $\sum a_nz^n$ une série entière de rayon $R>0$ et $f$ sa somme sur le disque $D(0,R)$.
Si $|f|$ admet un maximum local en zéro, alors $f$ est constante.
\end{theoreme}

\begin{theoreme}[Représentation intégrale des coefficients]
Soit $\sum a_nz^n$ une série entière de rayon $R>0$ et $f$ sa somme sur le disque $D(0,R)$.
Soit $r \in ]0,R[$.
Alors, pour tout $n\in\N$, on a 
\[a_n = \frac{1}{2\pi r^n}\int_0^{2\pi}f\left(re^{i\theta}\right)e^{-in\theta}d\theta. \]
\end{theoreme}

\begin{theoreme}[de Liouville, énoncé pour les séries entières]
Soit $\sum a_nz^n$ une série entière de rayon $+\infty$ et $f:\C\to\C$ sa somme.
Si $f$ est bornée, alors $f$ est constante.
\end{theoreme}
(Rappel de terminologie : $f$  bornée veut par définition dire $|f|$ bornée.)


\subsection{Fonctions analytiques}

\begin{prop}
Soit $\sum_{n\geq 0}a_nz^n$ une série entière de rayon de convergence $R>0$ et $f$ sa somme sur $D(0,R)$.
Alors $f$ est analytique sur $D(0,R)$.
\end{prop}

\begin{prop}
Une fonction analytique sur $U$ est holomorphe sur $U$.
\end{prop}


\begin{theoreme}[Zéros isolés pour les fonctions analytiques]
Soit $U\subseteq \C$ un ouvert connexe et $f$ une fonction analytique non identiquement nulle sur $U$.
Alors les zéros de $f$ sont isolés.
\end{theoreme}

\begin{theoreme}[Zéros isolés pour les fonctions analytiques]
Soit $U\subseteq \C$ un ouvert connexe,  $f:U\to\C$ analytique et $Z:=f^{-1}(\{0\})$ l'ensemble des zéros de $f$.
Si $Z$ admet un point d'accumulation dans $U$, alors $f$ est identiquement nulle sur $U$.
\end{theoreme}



\section{Intégration curviligne, primitives holomorphes}


\begin{prop}
Soit $U$ un ouvert de $\C$, $f\in \mathcal C^0(U,\C)$,  $\gamma:[0,1]\to U$ un chemin de $U$ de classe $\mathcal C^1$ et $\gamma^{opp}$ son chemin opposé.
Alors 
\[  \int_{\gamma^{opp}} f(z)dz = -\int_\gamma f(z)dz.\]
\end{prop}

\begin{prop}[Invariance par reparamétrage croissant]
Soit $U$ un ouvert de $\C$, $f\in \mathcal C^0(U,\C)$,  $\gamma:[0,1]\to U$ un chemin de $U$ de classe $\mathcal C^1$ et $\phi : [0,1]\to [01]$ une bijection croissante de classe $\mathcal C^1$.
Alors 
\[  \int_{\gamma\circ \phi } f(z)dz = \int_\gamma f(z)dz.\]
\end{prop}

\begin{prop}[Inégalité triangulaire]
Soit $U$ un ouvert de $\C$, $f\in \mathcal C^0(U,\C)$ et $\gamma:[0,1]\to U$ un chemin de $U$ de classe $\mathcal C^1$.
Alors :
\[ \left| \int_\gamma f(z)dz\right| \leq \int_0^1 \left|f(\gamma(t))\gamma'(t)\right|dt.\]
\end{prop}

\begin{prop}[Inégalité triangulaire, forme simplifiée, majoration brutale de $|f|$]
Soit $U$ un ouvert de $\C$, $f\in \mathcal C^0(U,\C)$ et $\gamma$ un chemin de $U$ de classe $\mathcal C^1_{pm}$.
Alors :
\[ \left| \int_\gamma f(z)dz\right| \leq \operatorname{long}(\gamma)\max_{z\in \supp \gamma}|f(z)|.\]
\end{prop}

\begin{proposition}
Soit $U\subseteq \C$ un ouvert, $f:U\to\C$  et $\gamma : [0,1]\to U$ un chemin $\mathcal C^1_{pm}$ dans $U$.

Si $f$ possède une primitive holomorphe $F$ sur $U$, alors  
\[\int_\gamma f(z)dz = F(\gamma(1)) - F(\gamma(0)).\]
\end{proposition}

% Après Goursat
%\begin{prop}
%Soit $f$ holomorphe sur un ouvert étoilé $U$.
%Alors $f$ possède une primitive holomorphe sur $U$.
%\end{prop}




\section{Théorème et formule(s) de Cauchy}

\begin{theoreme}[Lemme de Goursat]
Soit $U$ un ouvert de $\C$,  $f\in \mathcal O(U)$ et $T\subset U$ un triangle (enveloppe convexe de trois points non alignés), dont le bord est muni de son orientation canonique.
Alors $\int_{\partial T} f(z)dz=0$.
\end{theoreme}

\begin{prop}
Soit $U\subseteq \C$ un ouvert étoilé et $f\in \mathcal O(U)$.
Alors $f$ possède des primitives holomorphes sur $U$.
\end{prop}

\begin{theoreme}[Théorème intégral de Cauchy]
Soit $U\subset \C$ un ouvert, $f\in \mathcal O(U)$, et $K\subseteq U$ un compact à bord $\mathcal C^1$ par morceaux, dont le bord est muni de son orientation canonique.
Alors, $\int_{\partial K} f(z)dz = 0$.
\end{theoreme}





\begin{theoreme}[Formule de Cauchy]
Soit $U\subset \C$ un ouvert, $f\in \mathcal O(U)$, et $K\subseteq U$ un compact à bord $\mathcal C^1$ par morceaux, dont le bord est muni de son orientation canonique.
Soit $a\in K^{\circ}$.
Alors:
\[ f(a) = \frac{1}{2i\pi}\int_{\partial K} \frac{f(z)}{z-a}dz.\]
\end{theoreme}



\begin{theoreme}
Les fonctions holomorphes sont analytiques.
De plus, si $f\in \mathcal O(U)$ et $z_0 \in U$, le rayon de convergence du DSE de $f$ en $z_0$ a un rayon supérieur ou égal à $\operatorname{dist}(z_0,U^c)$.
\end{theoreme}

\begin{theoreme}
Soit $f$ une fonction holomorphe sur $U$.
Alors $f'$ est holomorphe sur $U$.
\end{theoreme}

\begin{theoreme}[de Morera]
Soit $U$ un ouvert de $\C$ et $f\in \mathcal C^0(U,\C)$ telle que pour tout triangle $T\subset U$, on ait $\int_{\partial T}f(z)dz=0$.
Alors, $f$ est holomorphe.
% on construit des primitives locales.
\end{theoreme}

\begin{theoreme}[Formule de la moyenne]
Soit $U\subseteq \C$ un ouvert,  $f \in \mathcal O(U)$ et $z_0\in U$.
Pour tout $r>0$ tel que $\overline{\D(z_0,r)} \subseteq U$, on a :
\[ f(z_0) = \frac{1}{2\pi}\int_0^{2\pi} f\left(z_0+re^{it}\right)dt\]
\end{theoreme}


\begin{theoreme}[Formules de Cauchy d'ordre supérieur]
Soit $U\subseteq \C$ un ouvert et $f\in \mathcal O(U)$.
Soit $K\subseteq U$ un compact à bords $C^1$ par morceaux et $z\in K^\circ$.
Alors, pour tout $n\geq 0$ :
\[ \frac{f^{(n)}(z)}{n!} = \frac{1}{2i\pi} \int_{\partial K} \frac{f(w)}{(w-z)^{n+1}}dw\]
\end{theoreme}

 
\begin{corollaire}[Inégalités de Cauchy pour cercles centrés]
Soit $U\subseteq \C$ un ouvert,  $f\in \mathcal O(U)$ et $\overline{\mathbb D(z,r)}$ un disque fermé inclus dans $U$.
Alors,  pour tout $n\geq 0$ :
\[ \left|f^{(n)}(z)\right|\leq \frac{n!}{r^n} \max_{w\in \mathcal C(z,r)} |f(w)|\]
\end{corollaire}

 \begin{theoreme}[de Liouville]
 Soit $f$ une fonction holomorphe sur $\C$.
 Si $f$ est bornée, alors $f$ est constante.
 \end{theoreme}
 

 
\begin{theoreme}[Principe du maximum]
Soit $U\subseteq \C$ un ouvert et $f \in \mathcal O(U)$.
Si $|f|$ admet un maximum local en un point $z_0$, alors $f$ est constante sur la composante connexe  de $U$ contenant $z_0$.
\end{theoreme}

\begin{corollaire}[du principe du maximum]
Soit $U\subseteq \C$ un ouvert,  $f \in \mathcal O(U)$ et $K\subseteq U$ un compact.
Alors $\max_{K} |f| = \max_{\partial K} |f|$.
\end{corollaire}

\begin{theoreme}[Lemme de Schwarz]
Soit $\D$ le disque unité, et $f :\D\to \D$ une fonction holomorphe telle que $f(0)=0$.
Alors:
\begin{enumerate}
\item $\forall z\in \D, |f(z)|\leq |z|$.
\item S'il existe $w \in \D^*$ tel que $|f(w)|= |w|$, ou bien si $|f'(0)|=1$, alors il existe $\lambda$ de module un tel que $\forall z\in \D, f(z)=\lambda z$.
\end{enumerate}
\end{theoreme}




\subsection{Étude locale des fonctions holomorphes}

\begin{theoreme}[Inversion locale pour les fonctions holomorphes]
Soit $U\subseteq \C$ un ouvert, $f\in \mathcal O(U)$ et $z_0\in U$ un point tel que $f'(z_0)\neq 0$.
Alors il existe un ouvert $V$ contenant $z_0$ tel que $W:=f(V)$ soit ouvert et que $f|_V$ soit un biholomorphisme de $V$ sur $W$.
\end{theoreme}

\begin{prop}[Étude aux points critiques des fonctions holomorphes]
Soit $U$ un ouvert connexe de $\C$, $f\in \mathcal O(U)$ une fonction non constante, $z_0\in U$ et $d = \min\{d\in \N^*,\: f^{(d)}(z_0)\neq 0\}$.
Alors il existe un voisinage ouvert $V\subseteq U$ de $z_0$, un voisinage ouvert  $W$ de $0$ et un biholomorphisme $\phi : V\to W$ tel que : 
\[ \forall z\in V, \: f(z) - f(z_0) = \phi(z)^d. \]
\end{prop}


\begin{theoreme}[de l'application ouverte pour les fonctions holomorphes]
Soit $U$ un ouvert connexe et $f\in \mathcal O(U)$ non constante.
Alors, $f$ est ouverte.
\end{theoreme}









\section{Séries de Laurent, singularités isolées, Résidus}

\subsection{Séries de Laurent}



\begin{theoreme}[Développement en série de Laurent des fonctions holomorphes sur une couronne]
Une fonction holomorphe sur une couronne $A(z_0,R,R')$ est développable en série de Laurent $\sum_{n\in\Z} a_n(z-z_0)^n$ sur cette couronne et pour tout $n\in\Z$ et $r \in ]R,R'[$ , on a :
\[a_n = \frac{1}{2\pi r^n}\int_0^{2\pi}f\left(re^{i\theta}\right)e^{-in\theta}d\theta. \]
\end{theoreme}



\subsection{Singularités isolées}



\begin{proposition}
Soit $z_0$ une singularités isolée d'une fonction holomorphe $f$.
Soit $\sum_{n\in \Z} a_n(z-z_0)^n$ le développement en série de Laurent de $f$ sur un disque épointé $\D(z_0,\epsilon)^*$.
Alors, $z_0$ est une singularité effaçable ssi $\forall n<0, a_n=0$.
\end{proposition}

%Idée de preuve : pour le sens difficile, si $f$ se prolonge en une fonction holomorphe $\tilde f$, alors on développe $\tilde f$ en série entière, et on invoque l'unicité des coefficients d'un développement en série de Laurent sur le disque épointé.

\begin{theoreme}[Théorème d'extension de Riemann]
Soit $z_0$ une singularité isolée d'une fonction holomorphe $f$.
Si $|f|$ est bornée au voisinage de $z_0$, alors la singularité est effaçable, autrement dit $f$ se prolonge holomorphiquement en $z_0$.
\end{theoreme}

\begin{proposition}[Comportement au voisinage d'un pôle]
Soit $f$ une fonction holomorphe admettant un pôle d'ordre $d$ en $z_0$.
Au voisinage de $z_0$, on a $f(z) \sim a_{-d}z^{-d}$.
\end{proposition}

\begin{theoreme}[Casoratti-Weiestrass]
Soit $z_0 \in U$ une singularité essentielle d'une fonction holomorphe $f \in \mathcal O(U\setminus \{z_0\})$.
Alors, pour tout voisinage $W$ de $z_0$ contenu dans $U$, l'image par $f$ de $W\setminus\{z_0\}$ est dense dans $\C$.
\end{theoreme}


\begin{proposition}
Soit $U$ un ouvert,  $Z\subset U$ un fermé constitué de points isolés et $f$ fonction holomorphe sur $U\setminus Z$.
Alors il y a équivalence entre :
\begin{itemize}
\item Pour tout $z_0\in Z$, $z_0$ n'est pas une singularité essentielle de $f$.
\item Au voisinage de tout $z_0\in Z$, $f$ s'écrit comme le quotient $g/h$ de deux fonctions holomorphes (définies sur un voisinage de $z_0$).
\end{itemize}
\end{proposition}


\subsection{Théorème des résidus}


\begin{theoreme}[des résidus]
Soit $U\subset \C$ un ouvert, $Z\subset U$ une partie discrète, $K\subset U$ un compact à bords $\mathcal C^1$ par morceaux avec $Z\cap \partial K = \emptyset$, et $f$ une fonction holomorphe sur $U\setminus Z$.
Alors:
\[ \int_{\partial K} f(z)dz = 2i\pi \sum_{\alpha \in K\cap Z} \operatorname{Rés}_\alpha(f).\]
\end{theoreme}



\subsection{Principe de l'argument et théorème de Rouché}



\begin{theoreme}[Principe de l'argument]
Soit $U\subseteq \C$ un ouvert, $f$ une fonction méromorphe à zéros isolés\footnote{Hypothèse rajoutée pour éviter que $f$ soit identiquement nulle sur une composante connexe de $U$. Demander que les zéros soient isolés permet de dire que $f$ admet un diviseur.} sur $U$ et $D = \sum_a m_a[a]$ son diviseur.
Si $K\subset U$ est un compact à bord $\mathcal C^1$ par morceaux, dont le bord ne contient aucun zéro ni pôle de $f$, alors :
\[ \frac{1}{2i\pi} \int_{\partial K} \frac{f'(z)}{f(z)}dz = \sum_{a \in K} m_a.\]
\end{theoreme}

%Idée de preuve : propriétés de la dérivée logarithmique.

\begin{theoreme}[Théorème de Rouché]
Soit $U\subset \C$ un ouvert, $f, g\in \mathcal O(U)$ et $K\subset U$ un compact.
Si $|f-g| < |f|$ sur $\partial K$, alors $f$ et $g$ ont même nombre de zéros dans $K$, comptés avec multiplicités.
\end{theoreme}

%Idée de preuve : on commence par se ramener à un compact à bords $\mathcal C^1$ par morceaux, puis on applique le principe de l'argument.



\begin{corollaire}[Localisation des racines des polynômes : version effective]
Soit $P=z^n + a_{n-1}z^{n-1}+\cdots + a_1z + a_0$ un polynôme de degré $n/\geq 1$.
Soit $R  = 2\max_{0\leq k\leq n-1} |a_k|^{\frac{1}{n-k}}$.
Alors $P$ admet $n$ racines dans le disque $\D(0,R)$, comptées avec multiplicités.
\end{corollaire}


Dans toute la suite du paragraphe, $U$ est un ouvert de $\C$, $(f_n)_n$ est une suite de fonctions holomorphes convergeant uniformément sur tout compact vers une fonction $f$.
Le théorème de Morera implique alors que $f$ est  holomorphe.

Pour simplifier l'énoncé des résultats, on suppose de plus $U$ \textbf{connexe}.

\begin{lemme}
Soit $K \subset U$ un compact. Si $f$ ne s'annule pas sur $\partial K$, alors pour $n$ grand, $f_n$ ne s'annule pas non plus sur $\partial K$ et $f$ et $f_n$ ont même nombre de zéros dans $K^\circ$, comptés avec multiplicités.
\end{lemme}

\begin{proposition}
Si les fonctions $f_n$ ne s'annulent pas dans $U$, alors soit $f$ ne s'annule pas sur $U$, soit $f$ est identiquement nulle.
\end{proposition}

\begin{proposition}
Si les fonctions $f_n$ sont injectives, alors soit $f$ est injective, soit $f$ est constante.
\end{proposition}


\end{document}








