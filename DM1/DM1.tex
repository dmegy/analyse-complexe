\documentclass[11pt,a4paper]{article}

%\usepackage[latin1]{inputenc} % ncessaire sur PC
\usepackage[utf8]{inputenc}

\usepackage{mathtools,amssymb,amsthm}
\usepackage{ifthen}
\usepackage{verbatim}
\newboolean{enonce} 
%\setboolean{enonce}{false}
\setboolean{enonce}{true}
\newboolean{indication} 
%\setboolean{indication}{true}
\setboolean{indication}{false}
%\usepackage{eepic}

\usepackage[dvips]{graphicx}

\usepackage[english,frenchb]{babel}
\usepackage{eurosym}
\usepackage{fancyhdr}

\usepackage{multicol,url}

\newcounter{encompt}
\newenvironment{exo}[1][]{\refstepcounter{encompt}\noindent{\textbf{Exercice~\theencompt}}~{\textbf{#1}}}{}
\newcommand{\refex}[1]{\thechapter.\ref{#1}}
\pagestyle{empty}
\newcommand{\ind}{ 1\hspace{-.55ex}\mbox{l}}
\setlength{\topmargin}{-0.5in}
\setlength{\textheight}{9.5in}
\setlength{\oddsidemargin}{0in}
\setlength{\evensidemargin}{0in}
\setlength{\textwidth}{6in}

\newcommand{\Card}{\mathrm{Card \,}}
\def\N{\mathbb N}
\def\R{\mathbb R}
\def\Sn{\frak{S}_n}

\begin{document}

\everymath{\displaystyle}

\pagestyle{fancy}

\title{DM de révisions\\ À rendre le mercredi 24 janvier\\ (lors du premier CM d'analyse complexe)}
\date{}
\maketitle

\begin{exo}
Pour chacune des contraintes ci-dessous, donner un exemple de série entière satisfaisant cette contrainte. 
\begin{enumerate}
\item Le rayon de convergence est $\sqrt[3]{2}$.
\item La série diverge sur tout le cercle d'incertitude\footnote{Souvent appelé \emph{cercle de convergence}.}.
\item La série converge sur tout le cercle d'incertitude.
\item Il existe un point du cercle d'incertitude où la série converge, et un autre point du cercle où la série diverge.
\item La série diverge en un seul point du cercle d'incertitude et converge en tous les autres.
\item (Bonus) La série diverge en exactement deux points du cercle d'incertitude et converge en tous les autres.
\end{enumerate}
\end{exo}

\begin{exo}
Déterminer le rayon de convergence de la série entière $\sum a_nz^n$  dans chacun des cas suivants :
\begin{multicols}{2}
\begin{enumerate}
\item  $a_n=2^n-3^n$; 
\item  $a_n=3^n+(-3)^n$;
\item $a_n=n\sin(n)$;
\item $a_n$ est la $n$-ème décimale de $e$;\columnbreak
\item  $a_n$ est le $n$-ème  nombre de Fibonacci.\footnote{Voir \url{https://oeis.org/A000045}}
\item $a_n=\frac{n!}{n^n}$
\end{enumerate}
\end{multicols}
\end{exo}
%
%\begin{exo}
%Déterminer le rayon de convergence des séries entières suivantes et calculer leur somme.
%\begin{enumerate}
%\item $\sum_{n\geq 0} \dfrac{z^n}{n^2+5n+6}$
%\item $\sum_{n\geq 0} n^2z^n$
%\item $\sum_{n\geq 0} \dfrac{z^n}{(n+2)n!}$
%\end{enumerate}
%\end{exo}

\begin{exo}
Développer en série entière autour de l'origine les fonctions suivantes.
Préciser le rayon de convergence de la série entière obtenue.
\begin{multicols}{2}
\begin{enumerate}
\item $f: \R\setminus\{-2,3\}\to\R, x\mapsto \dfrac{1}{x^2-x-6}$
\item $f: \R\to \R, x\mapsto \dfrac{1}{1+2x^2}$
%\item $f: \R\to\R, x\mapsto \dfrac{1}{x^2+x+1}$
%\item $f: \R\setminus\{2\}\to\R, x\mapsto \dfrac{1}{(x^2+1)(x-2)}$
\end{enumerate}
\end{multicols}
\end{exo}

\begin{exo}
Montrer que 
\[
\int_0^1 \frac{\ln(x)}{x+1}dx = \sum_{n\geq 1} \frac{(-1)^n}{n^2}
\]
[Tiré des annales récentes de l'UE Intégration-Probas. Petit ajout : montrer que cette quantité vaut $-\pi^2/12$. Ceci permet donc de calculer l'intégrale d'une fonction dont on ne connaît pas de primitive. À noter que ce genre d'astuce (ou l'utilisation d'intégrales doubles, ou d'intégrales à paramètres) ne marche pas toujours très bien. Dans le cours d'analyse complexe, nous verrons une méthode générale qui utilise une approche nouvelle, avec l'intégration curviligne dans le plan complexe. L'idée est de sortir de l'axe réel pour pouvoir contourner certains points. Les intégrales curvilignes sont ensuite calculées grâce à un théorème très puissant, le théorème des résidus.]

\end{exo}



\end{document}


