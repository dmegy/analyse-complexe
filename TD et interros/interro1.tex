\documentclass[11pt,a4paper]{article}

\usepackage[margin=2cm]{geometry}% gestion des marges etc
\usepackage[utf8]{inputenc} % caractères utf8 dans le fichier source
\usepackage[T1]{fontenc} % encodage en sortie
\usepackage[francais]{babel} % paramètres de langue : guillemets etc
\usepackage{amssymb,mathtools,amsthm}
\usepackage{stmaryrd,mathrsfs} % polices et symboles supplémentaires
\usepackage{mdframed,fancybox,graphicx}
\usepackage[dvipsnames]{xcolor}
\definecolor{preuve}{rgb}{0,0.2,0.5}

\usepackage{xypic,multicol,comment,variations,enumerate,enumitem,datetime,tasks}

\usepackage{hyperref}
\hypersetup{
    colorlinks=true,       % false: boxed links; true: colored links
    linkcolor=[rgb]{0.7,0.2,0.2},          % color of internal links
    citecolor=[rgb]{0.7,0.2,0.2},        % color of links to bibliography
    filecolor=[rgb]{0.7,0.2,0.2},      % color of file links
    urlcolor=[rgb]{0.7,0.2,0.2}           % color of external links
}

\usepackage{pgf,pgfmath,tikz}
\usetikzlibrary{arrows}
\usetikzlibrary[patterns]
\tikzset{every picture/.style={execute at begin picture={
   \shorthandoff{:;!?};}
}}




% - - - - - - -
% Spécifique à ce document :
%\usepackage{palatino, euler} % police : Palatino, et Euler pour les maths
\usepackage{fourier} % police de caractères : Adobe Utopia + Fourier math
\everymath{\displaystyle} % plus lisible mais casse l'homogénéité de la mise en page, tant pis la lisiblité passe en premier



\newcommand{\N}{\mathbb{N}}
\newcommand{\Z}{\mathbb{Z}}
\newcommand{\D}{\mathbb{D}}
\newcommand{\Q}{\mathbb{Q}}
\newcommand{\R}{\mathbb{R}}
\newcommand{\C}{\mathbb{C}}
\renewcommand{\H}{\mathbb{H}}
\newcommand{\K}{\mathbb{K}}
\renewcommand{\P}{\mathbb{P}}
\renewcommand{\S}{\mathbb{S}}
\newcommand{\B}{\mathbb{B}}
\newcommand{\U}{\mathbb{U}}

\DeclareMathOperator{\pgcd}{pgcd}
\DeclareMathOperator{\ppcm}{ppcm}
\DeclareMathOperator{\Id}{Id}
\DeclareMathOperator{\Bij}{Bij}
\DeclareMathOperator{\Fix}{Fix}
\DeclareMathOperator{\dist}{dist}
\DeclareMathOperator{\Card}{Card} % cardinal
\renewcommand{\Im}{\operatorname{Im}}
\renewcommand{\Re}{\operatorname{Ré}}
\renewcommand{\mid}{\;\ifnum\currentgrouptype=16 \middle\fi|\;}
\newcommand\eqdef{\mathrel{\overset{\makebox[0pt]{\mbox{\normalfont\tiny\sffamily déf}}}{=}}}
% égal par définition


\newcommand{\z}{\overline{z}}
\DeclareMathOperator{\PR}{\text{Ré}}
\DeclareMathOperator{\PI}{\text{Im}}
\DeclareMathOperator{\Log}{\text{Log}}
\newcommand{\vecteur}[2]{\begin{pmatrix}#1 \\ #2 \end{pmatrix}}
\DeclareMathOperator{\del}{\frac{\partial}{\partial z}}
\DeclareMathOperator{\delbarre}{\frac{\partial}{\partial \z}}



\newcommand{\ensemble}[2]{\left \{ #1  
    \ifx&#2&%
       %
    \else%
       \, \middle | \, #2%
    \fi%
\right \}}

\newcommand{\modulo}[1]{\:\left(\operatorname{mod}#1\right)}

% Environnements : 

\theoremstyle{definition}
\newtheorem{theoreme}{Th\'eor\`eme}[section]
\newtheorem{proposition}[theoreme]{Proposition}
\newtheorem{corollaire}[theoreme]{Corollaire}
\newtheorem{lemme}[theoreme]{Lemme}
\renewenvironment{proof}{\color{preuve}\emph{Démonstration.~}}{\qed}
\newenvironment{red}{\begin{quote}\color{preuve}\emph{Exemple de rédaction:}\\}{\end{quote}}

\newtheorem{propdef}[theoreme]{Proposition et Définition}
\newtheorem{axiomedef}[theoreme]{Axiome et Définition}
\newtheorem{definition}[theoreme]{D\'efinition}
\newtheorem{vocabulaire}[theoreme]{Vocabulaire}
\newtheorem{exercice}[theoreme]{Exercice}

\newtheorem{exemple}[theoreme]{Exemple}
\newtheorem{exemples}[theoreme]{Exemples}
\newtheorem{attention}[theoreme]{Mise en garde}


\newtheorem{ex}{Exercice}

\theoremstyle{plain}
\newtheorem{remarque}[theoreme]{Remarque}
\newtheorem{methode}[theoreme]{Méthode}




%%%%%%%%%%
% MACROS POUR REMPLACER ANSWERS

%\newenvironment{exo}{\begin{ex}}{\end{ex}}
%\excludecomment{hint} % on n'affiche pas les hints
%\excludecomment{sol} % on n'affiche pas les solutions




% - - - - - - - - - - - - - -
% PARAMETRAGE DU PACKAGE ANSWERS 
% POUR LES INDICATIONS ET CORRECTIONS
% - - - - - - - - - - - - - - 

\usepackage{answers}
\Newassociation{sol}{Soln}{solutions}
% ira dans le fichier d'identifiant 'solutions' et nom de fichier \jobname_sol.tex
% et écrira les solutions dans un environnement 'Soln'

\newenvironment{exo}{\begin{ex}\label{enonce.\theex} }{\end{ex} }

\renewenvironment{Soln}[1]{\noindent{\bf Correction de l'exercice \ref{enonce.#1}.} }
% - - - - - - - - - - - - - - 
% FIN  PARAMETRAGE ANSWERS
% - - - - - - - - - - - - - - 


%\usepackage{tasks}
%%%%%%%%%%%%%%%%%%%

\newcommand{\z}{\overline{z}}
\DeclareMathOperator{\PR}{\text{Ré}}
\DeclareMathOperator{\PI}{\text{Im}}
\DeclareMathOperator{\Log}{\text{Log}}
\newcommand{\vecteur}[2]{\begin{pmatrix}#1 \\ #2 \end{pmatrix}}

%\pagestyle{empty}


\begin{document}

%%%%%%%%%%%%%%%%%%%%%%%%%%%%%%%%%%%%%%

%%%%%%%%%%%%%%%%%%%%%%%%%%%%%%%%%%%%%%

%%%%%%%%%%%%%%%%%%%%%%%%%%%%%%%%%%%%%%
\Opensolutionfile{indications}[\jobname_hints]
\Opensolutionfile{solutions}[\jobname_sol]



\newpage
\noindent \textbf{\textsf{Université de Lorraine \hfill Analyse complexe}}
\smallskip
\noindent\rule{\textwidth}{2pt}
\begin{center}
{\huge \textbf{Interrogation 1}}
\end{center}
\noindent\rule{\textwidth}{2pt}

Dans les énoncés, lorsqu'on écrit $z:=x+iy$, il est sous-entendu que $x$ et $y$ sont des réels.


\begin{exo}
Soit $U$ un ouvert de $\C$ et $z_0\in U$.
\begin{enumerate}
\item Soit $f : U\to \C$. Définir les assertions \og $f$ est $\C$-dérivable en $z_0$ \fg{} et \og $f$ est holomorphe sur $U$.\fg
\item Soit $f : U\to \C$ différentiable.  Définir l'assertion \og $f$ vérifie les conditions de Cauchy-Riemann en $z_0$.\fg{}
%\item Bonus : donner un exemple de fonction $g : \C\to \C$ qui est $\C$-dérivable en l'origine mais qui n'est pas holomorphe sur $\C$.
\end{enumerate}
\begin{sol}
\begin{enumerate}
\item La fonction $f$ est dérivable au sens complexe en $z_0$ si le taux d'accroissement $\frac{f(z)-f(z_0)}{z-z_0}$ admet une limite (finie) lorsque $z$ tend vers $z_0$ dans $U$. 
%Si cette limite existe, on l'appelle le nombre dérivé de $f$ en $z_0$ et on le note $f'(z_0)$. 
La fonction $f$ est holomorphe sur $U$ si elle est dérivable au sens complexe en tout point de $U$.
\item Notons $u$ et $v$ les parties réelle et imaginaire de $f$.  Alors, $f$ vérifie les conditions de Cauchy-Riemann en $z_0$ si $\partial_x u (z_0) = \partial_y v (z_0)$ et $\partial_y u (z_0) = -\partial_x v(z_0)$. Ceci est équivalent au fait que la matrice jacobienne de $f$ en $z_0$ soit une matrice $\C$-linéaire, c'est-à-dire de la forme $\begin{pmatrix}
a& -b \\ b & a
\end{pmatrix}$, où $a$ et $b$ sont réels.
\end{enumerate}
\end{sol}

\end{exo}

\begin{exo}
On considère la fonction $f : \C\setminus \{z\in\C, \PR(z)\PI(z)=0\}\to \C, z\mapsto \dfrac{\overline{z}}{\PR(z)\PI(z)}$.
\begin{enumerate}
\item Écrire $f$ comme une fonction d'un ouvert de $\R^2$ dans $\R^2$ et calculer sa matrice jacobienne en tout point.
\item Déterminer l'ensemble des points du plan où $f$ est $\C$-dérivable.
\item Existe-t-il des ouverts du plan sur lesquels $f$ est holomorphe et si oui lesquels ?
\end{enumerate}

\begin{sol}
\begin{enumerate}
\item On obtient $f : \R^2 \setminus \{(x,y)\in\R^2\:|\:xy=0\} \to \R^2, \vecteur{x}{y} \mapsto \vecteur{1/y}{-1/x}$,  donc
$ \operatorname{Jac}(f) = \begin{pmatrix}
0 & -1/y^2 \\
1/x^2 & 0
\end{pmatrix}
$.
\item La fonction $f$ étant différentiable, elle est donc dérivable au sens complexe exactement là où les conditions de Cauchy-Riemann sont vérifiées, c'est-à-dire aux points $(x,y) \in U$ pour lesquels $\frac{1}{x^2} = \frac{1}{y^2}$, c'est-à-dire $x^2=y^2$, c'est-à-dire $x=y$ ou $x=-y$. Il s'agit de l'union de deux droites.
\item Le lieu où $f$ est dérivable au sens complexe est d'intérieur vide. Il n'existe donc pas d'ouvert de $U$ sur lequel la fonction $f$ est holomorphe.

\end{enumerate}
\end{sol}
\end{exo}

%\begin{exo}
%Déterminer une condition nécessaire et suffisante sur $\lambda\in\R$ pour que la fonction
%\[ f_\lambda : \C\to \C,  x+iy \mapsto e^y(\cos x + i\lambda \sin x)\]
%soit holomorphe sur $\C$.
%\begin{sol}
%\end{sol}
%\end{exo}
%
%\begin{exo}
%Soit $u : \C\to \R,  x+iy \mapsto e^{x^2-y^2}\sin(2xy)$. 
%Déterminer toutes les fonctions $v : \C\to \R$ telles que $u+iv$ soit une fonction holomorphe sur $\C$. 
%(Attention c'est bien un sinus et non pas un cosinus.)
%\begin{sol}
%\end{sol}
%\end{exo}

\begin{exo}
Soit $u : \C\to \R,  x+iy \mapsto e^y\cos x$. 
Déterminer toutes les fonctions $v : \C\to \R$ telles que $f:=u+iv$ soit une fonction holomorphe sur $\C$. 
Écrire alors $f$ en fonction de la variable $z$ en donnant une expression simplifiée.
\begin{sol}
Soit $v : \C\to \R$ différentiable et soit $f=u+iv$. 
Si $f$ est holomorphe sur $U$ , alors $f$ satisfait les conditions de Cauchy-Riemann en tout point.
Ceci permet d'obtenir les deux dérivées partielles de $v$ grâce à celles de $u$, qui sont connues :
\[ \partial_x v = -\partial_y u = -e^y\cos(x) \text{ et } \partial_y v = \partial_x u = -e^y\sin(x).\]
Fixons $x\in \R$.  La fonction partielle $\R\to \R, y\mapsto v(x,y)$ a pour dérivée $y\mapsto \partial_y v = \partial_x u = -e^y\sin(x)$.  On en déduit que la fonction $\R\to\R, y\mapsto  \partial_y v + e^y\sin(x)$ a une dérivée nulle, donc est localement constante sur $\R$ qui est connexe, donc est constante sur $\R$. On note $h(x)$ cette constante, qui dépend du réel $x$ qui a été fixé en début de paragraphe. 
On a donc $v(x,y) = -e^y\sin(x) +h(x)$.


Faisons maintenant varier $x$. Comme $f$ est différentiable, la fonction $x\mapsto h(x)$ est dérivable, en la variable $x$ et en dérivant suivant $x$ on obtient $\partial_xv (x,y) = -e^y\cos(x)+h'(x)$.

Or, d'après les relations de Cauchy-Riemann rappelées plus haut, on sait que $ \partial_x v = -\partial_y u = -e^y\cos(x)$, d'où l'on tire que $h$ a une dérivée nulle, donc est localement constante. Comme $\C$ est connexe, $h$ est constante, égale à un réel $K$.

Finalement,  si $f=u+iv$ est holomorphe, on a prouvé qu'il existe $K\in \R$ tel que $v(x,y)=-e^y\sin(x)+K$.

Réciproquement, si $K$ est un réel et $v(x,y):=-e^y\sin(x)+K$, alors la fonction $f:=u+iv$ est différentiable et vérifie les conditions de Cauchy-Riemann en tout point de $\C$, donc est holomorphe sur $\C$.

Enfin, on voit que $f(z)=e^y(\cos x-i\sin x)+iK = e^ye^{-ix}+iK = e^{y-ix}+iK = e^{-iz}+iK$. 
Si on ne voit pas directement que $y-ix=-iz$, on utilise la méthodologie de base :
$y-ix = \dfrac{z-\bar z}{2i} - i \dfrac{z+\bar z}{2} 
=\dfrac{z-\bar z}{2i} + \dfrac{z+\bar z}{2i}
=  \dfrac{2z}{2i}= -iz$.
\end{sol}
\end{exo}

\begin{exo}
Rappeler la définition des notations de Wirtinger $\dfrac{\partial}{\partial z}$ et $\dfrac{\partial}{\partial \overline z}$, puis calculer $\dfrac{\partial}{\partial \overline z}\left(\dfrac{z+\bar z}{z\bar z}\right)$.
\begin{sol}
D'après le cours, les opérateurs $\dfrac{\partial}{\partial z}$ et $\dfrac{\partial}{\partial \overline z}$ sont définis par 
$\dfrac{\partial}{\partial z} = \frac12 \left(\frac{\partial}{\partial x}-i\frac{\partial}{\partial y}\right)$ et$\dfrac{\partial}{\partial \overline z} = \frac12 \left(\frac{\partial}{\partial x}+i\frac{\partial}{\partial y}\right)$.

On a  
\[ 
\dfrac{\partial}{\partial \overline z}\dfrac{z+\bar z}{z\bar z}
=\dfrac{\partial}{\partial \overline z}\left(\dfrac{1}{\bar z}\right)+\dfrac{\partial}{\partial \overline z}\left(\dfrac{1}{z}\right)
= -\dfrac{1}{\bar z^2}+0 
=-\dfrac{1}{\bar z^2} \]
Si on veut justifier le calcul de $\dfrac{\partial}{\partial \overline z}\left(\dfrac{1}{\bar z}\right)$, on peut par exemple utiliser la formule du produit appliquée à $1=\bar z \times \dfrac{1}{\bar z}$ ce qui donne $0 = \dfrac{\partial}{\partial \overline z} (1) = \dfrac{\partial}{\partial \overline z} (\overline z) \times \dfrac{1}{\bar z} + \bar z \times \dfrac{\partial}{\partial \overline z} \left(\dfrac{1}{\bar z}\right)$. On peut bien sûr tout écrire en fonction de $x$ et $y$ et revenir à la définition.
\end{sol}
\end{exo}


%\begin{exo}
%\begin{enumerate}
%\item  Donner un exemple de fonction $f : \C\to \C$ qui est $\C$-dérivable en l'origine mais qui n'est pas holomorphe sur $\C$.
%\item Donner un exemple de fonction $g : \C\to \C$ admettant des dérivées partielles en l'origine, dont les dérivées partielles en l'origine satisfont les conditions de Cauchy-Riemann, mais qui n'est pas dérivable au sens complexe en l'origine.
%\end{enumerate}
%\begin{sol}
%\end{sol}
%\end{exo}

\Closesolutionfile{indications}
\Closesolutionfile{solutions}


%\newpage
%\section{Indications pour les exercices}

%\Readsolutionfile{indications}

\newpage
 %commenter pour ne pas afficher les solutions
\section*{Correction de l'interrogation 1 (signaler d'éventuelles fautes!)}
\Readsolutionfile{solutions}


\end{document}
